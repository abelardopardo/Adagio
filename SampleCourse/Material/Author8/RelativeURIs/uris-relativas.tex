% $Id: uris-relativas.tex,v 1.1 2008-01-14 18:15:39 jaf Exp $

\part{URIs relativas y absolutas}


%---------------------------------------------------------------------- SLIDE -
\begin{slide}{Introducción}
  \begin{itemize}
  \item En (X)HTML, en un hiperenlace, imagen, etc. es necesario
    especificar una URI.
  \item El navegador necesita la URI completa para seguir el 
    hiperenlace, cargar la imagen, etc.
  \item Una URI se puede especificar como:
    \begin{itemize}
    \item URI absoluta.
    \item URI relativa a un servidor.
    \item URI relativa.
    \end{itemize}
  \end{itemize}
\end{slide}
%------------------------------------------------------------------------------

%---------------------------------------------------------------------- SLIDE -
\begin{slide}{URI absoluta}
  \begin{itemize}
  \item Se especifica directamente la URI completa del recurso.
  \item En HTTP, incluye el identificador de protocolo, servidor,
    ruta en el servidor y parámetros.
  \item El navegador simplemente toma la URI.
\end{itemize}

\begin{Verbatim}[fontfamily=tt,fontsize=\fontsize{8}{8}]
<a href="http://www.it.uc3m.es/labttlat/lab8/">...</a>
\end{Verbatim}

\end{slide}
%------------------------------------------------------------------------------

%---------------------------------------------------------------------- SLIDE -
\begin{slide}{URI relativa al servidor}
  \begin{itemize}
  \item Se especifica de forma absoluta la ruta del recurso
    (comenzando por ``/''), pero no
    se indica protocolo ni servidor.
  \item El navegador toma el protocolo y servidor del recurso
    en el cual está el enlace, imagen, etc.
\end{itemize}

\begin{Verbatim}[fontfamily=tt,fontsize=\fontsize{8}{8}]
<a href="/labttlat/lab8/">...</a>
\end{Verbatim}

\end{slide}
%------------------------------------------------------------------------------

%---------------------------------------------------------------------- SLIDE -
\begin{slide}{URI relativa}
  \begin{itemize}
  \item Se especifica sólo la ruta del recurso relativa (no comienza
    por ``/''), pero no
    se indica protocolo ni servidor, ni parte inicial de la ruta.
  \item El navegador toma el protocolo, servidor y parte inicial de la
    ruta del recurso
    en el cual está el enlace, imagen, etc.
    \begin{itemize}
    \item Para calcular la ruta, se toma la ruta del recurso actual
      excepto su último nivel (similar a la forma de nombrar ficheros
      en un sistema de ficheros).
    \end{itemize}
  \end{itemize}

\begin{Verbatim}[fontfamily=tt,fontsize=\fontsize{8}{8}]
<a href="lab8/">...</a>
\end{Verbatim}



\end{slide}
%------------------------------------------------------------------------------

%---------------------------------------------------------------------- SLIDE -
\begin{slide}{Ejemplo: URIs relativas}

\begin{center}
  \resizebox{0.9\textwidth}{!}{
    \includegraphics{uris-relativas.eps}}
\end{center}

\end{slide}
%------------------------------------------------------------------------------

%---------------------------------------------------------------------- SLIDE -
\begin{slide}{Recomendaciones de diseño}

  \begin{itemize}
  \item Es recomendable utilizar rutas relativas siempre que sea
    posible:
    \begin{itemize}
    \item Se puede cambiar la aplicación de servidor o ruta sin
      necesidad de cambiar ninguna URI en los servlets, JSP, (X)HTML,
      etc.
    \end{itemize}
  \end{itemize}

\end{slide}
%------------------------------------------------------------------------------




%%% Local Variables: 
%%% mode: latex
%%% TeX-master: t
%%% End: 
