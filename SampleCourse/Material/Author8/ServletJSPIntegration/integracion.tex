% $Id: integracion.tex,v 1.1 2008-01-14 18:15:38 jaf Exp $

\part{Integración de Servlets y JSP}


%---------------------------------------------------------------------- SLIDE -
\begin{slide}{Introducción}
  \begin{itemize}
  \item Una aplicación Web realiza tareas de procesado y presentación:
    \begin{itemize}
    \item Los Servlets son adecuados para procesado.
    \item Las páginas JSP son adecuadas presentación.
    \end{itemize}
  \item Una aplicación Web puede combinar Servlets y páginas JSP:
    \begin{itemize}
    \item Procesado de parámetros de la petición: Servlets.
    \item Acceso a bases de datos: Servlets.
    \item Lógica de la aplicación: Servlets.
    \item Presentación (vistas): JSP.
    \end{itemize}
  \end{itemize}
\end{slide}
%------------------------------------------------------------------------------

%---------------------------------------------------------------------- SLIDE -
\begin{slide}{Modelo de funcionamiento (I)}
  \begin{enumerate}
  \item El cliente envía la petición HTTP a un servlet.
  \item El servlet procesa la petición.
    \begin{itemize}
    \item Si es necesario, se conecta a la base de datos.
    \end{itemize}
  \item El servlet redirige la petición a un JSP.
    \begin{itemize}
    \item Si es necesario, añade \emph{beans} como parámetros.
    \end{itemize}
  \item El JSP lee los parámetros y devuelve la respuesta 
    formateada visualmente al usuario.
  \end{enumerate}
\end{slide}
%------------------------------------------------------------------------------

%---------------------------------------------------------------------- SLIDE -
\begin{slide}{Modelo de funcionamiento (II)}

\begin{center}
  \resizebox{0.9\textwidth}{!}{
    \includegraphics{mvc.eps}}
\end{center}

\end{slide}
%------------------------------------------------------------------------------

%---------------------------------------------------------------------- SLIDE -
\begin{slide}{Mecanismos de redirección de peticiones}

  \begin{itemize}
  \item Hay dos formas de redirigir una petición a otro recurso:
    \begin{itemize}
    \item Redirecciones HTTP  (\emph{sendRedirect}):
      \begin{itemize}
      \item El servidor envía una respuesta al 
        cliente con un código 3xx y la URI a la que este debe 
        enviar la petición.
      \end{itemize}
    \item Redirecciones internas en el servidor (\emph{forward}):
      \begin{itemize}
      \item Se redirige la petición de un recurso a otro dentro
        de la misma aplicación Web.
      \item El recurso de la última redirección devuelve al cliente
        la respuesta HTTP.
      \item La redirección es transparente para el cliente.
      \end{itemize}
    \end{itemize}
  \end{itemize}
\end{slide}
%------------------------------------------------------------------------------

%---------------------------------------------------------------------- SLIDE -
\begin{slide}{Redirecciones \emph{sendRedirect}}
  \begin{itemize}
  \item Fuerza el envío de una respuesta HTTP de redirección al cliente.
  \item El cliente envía la petición a la URI recibida en la respuesta.
  \end{itemize}

\begin{Verbatim}[fontfamily=tt,fontsize=\fontsize{8}{8}]
// Redirección con URI absoluta
response.sendRedirect("http://www.ejemplo.es/");

// Redirección con URI relativa a la URI de la petición actual
response.sendRedirect("otra.html");
\end{Verbatim}
\end{slide}
%------------------------------------------------------------------------------

%---------------------------------------------------------------------- SLIDE -
\begin{slide}{Redirecciones \emph{forward}}
  \begin{itemize}
  \item Un Servlet o JSP reenvía la petición a otro recurso (Servlet,
    JSP, HTML) de la
    misma aplicación Web.
  \item El cliente no se entera de la redirección (p.e., el navegador
    muestra la URI original de la petición, no la redirigida).
  \item El control retorna al finalizar el método \emph{forward},
    por lo que conviene que sea lo último que se ejecuta.
  \end{itemize}

\end{slide}
%------------------------------------------------------------------------------

%---------------------------------------------------------------------- SLIDE -
\begin{slide}{Redirecciones \emph{forward}}
  \begin{itemize}
  \item \emph{Forward} desde un Servlet:

\begin{Verbatim}[fontfamily=tt,fontsize=\fontsize{8}{8}]
RequestDispatcher rd = request.getRequestDispatcher("/vista.jsp");
rd.forward(request, response);
\end{Verbatim}

\item \emph{Forward} desde un JSP:

\begin{Verbatim}[fontfamily=tt,fontsize=\fontsize{8}{8}]
<jsp:forward page="/vista.jsp"/>
\end{Verbatim}

\end{itemize}
\end{slide}
%------------------------------------------------------------------------------

%---------------------------------------------------------------------- SLIDE -
\begin{slide}{Redirecciones \emph{forward} con parámetros}
  \begin{itemize}
  \item El objeto de la petición (\emph{ServletRequest}) de los recursos
   origen y destino de la redirección es el mismo:
   \begin{itemize}
   \item Se pueden añadir parámetros como atributos a la petición.
   \end{itemize}
 \end{itemize}
\begin{Verbatim}[fontfamily=tt,fontsize=\fontsize{8}{8}]
Noticia nuevaNoticia = (...)
request.setAttribute("noticia", nuevaNoticia);
RequestDispatcher rd = request.getRequestDispatcher("/vista.jsp");
rd.forward(request, response);
\end{Verbatim}

\end{slide}
%------------------------------------------------------------------------------

%---------------------------------------------------------------------- SLIDE -
\begin{slide}{Redirecciones \emph{forward} con parámetros}
  \begin{itemize}
  \item Recogida de parámetros desde un Servlet:
\begin{Verbatim}[fontfamily=tt,fontsize=\fontsize{8}{8}]
Noticia nuevaNoticia = (Noticia) request.getAttribute("noticia");
\end{Verbatim}
\item Recogida de parámetros desde un JSP:
\begin{Verbatim}[fontfamily=tt,fontsize=\fontsize{8}{8}]
<jsp:useBean id="noticia" class="beans.Noticia" 
             scope="request" />
\end{Verbatim}

  \end{itemize}
\end{slide}
%------------------------------------------------------------------------------





%%% Local Variables: 
%%% mode: latex
%%% TeX-master: t
%%% End: 
