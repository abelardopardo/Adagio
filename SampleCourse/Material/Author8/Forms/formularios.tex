% $Id: formularios.tex,v 1.1 2008-01-14 18:15:38 jaf Exp $

\part{Envío de parámetros de formularios}


%---------------------------------------------------------------------- SLIDE -
\begin{slide}{Envío de parámetros de formularios}
  \begin{itemize}
  \item El envío depende del método HTTP y la
    codificación:
    \begin{itemize}
    \item Método HTTP:
      \begin{itemize}
      \item Método GET.
      \item Método POST.
      \end{itemize}
    \item Codificación:
      \begin{itemize}
      \item \verb+application/x-www-form-urlencoded+
      \item \verb+multipart/form-data+
      \end{itemize}
    \end{itemize}
  \end{itemize}
\end{slide}
%------------------------------------------------------------------------------

%---------------------------------------------------------------------- SLIDE -
\begin{slide}{Envío de parámetros de formularios}
  \begin{itemize}
  \item Codificación URL--encoded:
    \begin{itemize}
    \item Lista de parámetros separados por ``\&''.
    \item Para cada parámetro se especifica nombre ``='' valor.
    \item Los caracteres especiales (no letras/dígitos ASCII) 
      se codifican en hexadecimal por su código UTF-8.
    \item Con método GET o POST.
    \item No se usa para campos de tipo \emph{file}.
    \end{itemize}
  \end{itemize}

\begin{Verbatim}[fontfamily=tt,fontsize=\fontsize{8}{8}]
usuario=juan&clave=juanpw&ssid=7fgxc&enviar=enviar
nombre=juan%20l%C3%B3pez%20l%C3%B3pez
\end{Verbatim}


\end{slide}
%------------------------------------------------------------------------------

%---------------------------------------------------------------------- SLIDE -
\begin{slide}{Envío de parámetros de formularios}
  \begin{itemize}
  \item Codificación URL--encoded con GET:
    \begin{itemize}
    \item Los parámetros se codifican en la ruta (\emph{path})
      de la petición HTTP.
    \item Sólo apto para operaciones idempotentes.
    \end{itemize}
  \end{itemize}

\begin{Verbatim}[fontfamily=tt,fontsize=\fontsize{8}{8}]
GET /jaf/cgi-bin/html2xhtml.cgi?tipo=auto&html=default.html HTTP/1.1
Host: www.ejemplo.es
(...)
\end{Verbatim}


\end{slide}
%------------------------------------------------------------------------------

%---------------------------------------------------------------------- SLIDE -
\begin{slide}{Envío de parámetros de formularios}
  \begin{itemize}
  \item Codificación URL--encoded con POST:
    \begin{itemize}
    \item Los parámetros se codifican en el cuerpo
      de la petición HTTP.
    \end{itemize}
  \end{itemize}

\begin{Verbatim}[fontfamily=tt,fontsize=\fontsize{8}{8}]
POST /jaf/cgi-bin/html2xhtml.cgi HTTP/1.1
(...)
Content-Length: 27
Content-Type: application/x-www-form-urlencoded

tipo=auto&html=default.html
\end{Verbatim}


\end{slide}
%------------------------------------------------------------------------------

%---------------------------------------------------------------------- SLIDE -
\begin{slide}{Envío de parámetros de formularios}
  \begin{itemize}
  \item Codificación Multipart (RFC 2388):
    \begin{itemize}
    \item Datos encapsulados con un mensaje multiparte MIME.
    \item Sólo con método POST.
    \item Necesario para enviar campos de tipo \emph{file}.
    \item No compatible con
      \emph{HttpServletRequest.getParameter(...)}
      \begin{itemize}
      \item Es necesario utilizar APIs adicionales desde un Servlet/JSP.
      \end{itemize}
    \end{itemize}
  \end{itemize}
\end{slide}
%------------------------------------------------------------------------------

%---------------------------------------------------------------------- SLIDE -
\begin{slide}{Ejemplo: multipart/form--data}


\begin{Verbatim}[fontfamily=tt,fontsize=\fontsize{7}{7}]
POST /jaf/cgi-bin/html2xhtml.cgi HTTP/1.1
(...)
Content-Type: multipart/form-data; boundary=----------2qYzCGdatrpobJh4m5rz50
Content-Length: 972

------------2qYzCGdatrpobJh4m5rz50
Content-Disposition: form-data; name="tipo"

auto
------------2qYzCGdatrpobJh4m5rz50
Content-Disposition: form-data; name="html"; filename="readme.html"
Content-Type: text/html

<html xmlns="http://www.w3.org/1999/xhtml">
(...)
</html>
------------2qYzCGdatrpobJh4m5rz50--
\end{Verbatim}

\end{slide}
%------------------------------------------------------------------------------


%%% Local Variables: 
%%% mode: latex
%%% TeX-master: t
%%% End: 
